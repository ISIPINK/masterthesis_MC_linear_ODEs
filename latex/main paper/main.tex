\documentclass[12pt]{article}
\usepackage{amsmath,amssymb,amsthm}
\usepackage{comment}

% Customize page margins
\usepackage[top=1in,bottom=1in,left=1.5in,right=1in]{geometry}

% Set line spacing to double
\usepackage{setspace}
\doublespacing

\newtheoremstyle{break}{4pt}{4pt}{}{}{\bfseries}{\vspace{2 pt}}{\newline}{}
\theoremstyle{break}
\newtheorem{theorem}{Theorem}[subsection]
\newtheorem{example}[theorem]{Example}
\newtheorem{definition}[theorem]{Definition}
\newtheorem{notation}[theorem]{Notation}
\newtheorem{lemma}[theorem]{Lemma}
\newtheorem{conjecture}[theorem]{Conjecture}
\newtheorem{corollary}[theorem]{Corollary}
\newtheorem{technique}[theorem]{Technique}
\newtheorem{pythonn}[theorem]{Python Code}
\newtheorem{related}[theorem]{Related Work}


% Define command for bold vectors
\newcommand{\vect}[1]{\boldsymbol{#1}}

% Begin document
\begin{document}

% Title page
\title{Unbiased Monte Carlo for Recursive Integrals}
\author{Isidoor Pinillo Esquivel}
\maketitle


\begin{abstract}
    We will write this at the end.
\end{abstract}

\tableofcontents

\section{Introduction}

Follows $y'=y$ example + modifications
This introduces recursive Monte Carlo and important Monte Carlo techniques used extensively later.

\section{Essentials}

\begin{example}[non-linearity]
    $y'=y^{2}$
\end{example}

\begin{example}[analytic functions]
    $e^{E[X]}$
\end{example}

\begin{example}[half variance phenomenon]
    randomized trapezium rule for introduction of the half variance phenomenon and an argument
    by integrating a polynomial for intuition
\end{example}

\begin{technique}[tail recursion]
    discuss problems with implementing recursion and solutions
\end{technique}

\begin{example}[coupled recursion]
    example with $y'=y$
\end{example}

\begin{example}[recursion in recursion]
    maybe induction in induction proof example
\end{example}

\section{$1$D Recursive Integrals}

\begin{example}[general linear recursive integral]
    We have algo in mind for this case based on coupled recursion on disjunct sets.
\end{example}

\begin{theorem}[green functions]
    green function stuff that we will be needing, we aren't sure in how much detail we're going to go.
\end{theorem}

\begin{example}[IVP]
    An IVP example probably using DRRMC maybe compare it to parareal. Maybe also non-linear algo
\end{example}

\begin{example}[BVP]
    A BVP example using yet another algo that hopefully has the half variance phenomenon.
\end{example}

\section{Higher D Recursive Integrals}

\begin{example}[problems with geometry]
    $2$D integral that is difficult because of its geometry
\end{example}

\begin{example}[numerical green functions]
    There will be probably some green functions that we need
    that don't have an analytic expression yet.
\end{example}

\begin{example}[recursive Brownian motion]
    WoS like way to simulate Brownian motion wich is related to the green function
    of the heat equation
\end{example}

\begin{example}[heat equation]
    a geometric robust way to solve the heat equation and a higher order method to solve
    the heat equation
\end{example}

\section{Appendix}
Derivation of the green functions and some expressions.

\end{document}
